\chapter{Luminescent Organic Materials}
\label{chapter: luminscent organic materials}
%%%%%%%%%%%%%%%%%%%%%%%%%%%%%%
\section{Introduction}\label{section: lom introduction}
%%%%%%%%%%%%%%%%%%%%%%%%%%%%%%
Luminescent organic molecules are used in numerous biological and technological applications. In aqueous solution they are employed extensively in biological imaging, probing, and detection. Deposited as thin films and aggregates, they represent the next generation of organic optoelectronics, where availability and low cost of starting materials, straightforward syntheses, and lightweight, flexible devices are attractive features. Perhaps most importantly, the luminescent response of molecular organic systems can be tuned with relative ease compared to their inorganic counterparts.

Since the discovery of electroluminsecence in the 1960s, intensive efforts in academia and industry have delivered considerable progress in the field of organic electronics, leading to  the development of applications such as field-effect transistors, photovoltaic cells, optical memory devices, and single-crystal lasers.\cite{Ostroverkhova2016} The most prominent success story are certainly \acp{OLED}, which have already reached market adoption for lighting and display purposes. However, in many areas organic systems suffer from low efficiencies, trial-and-error optimisation, and decreased performance in aggregated form versus solution. 
To advance, there must be control over both the supermolecular structure of the material and the electronic structure of the  molecules within. Neither of property exists in isolation, and their interplay must also be understood, which somewhat complicates matters. Of these contributing factors, it is the electronic structure and its relationship with the environment which are of interest in this work. The luminescent response of molecules can change drastically from one medium to another, whether in the gas phase, as a solution, aggregated as clusters, or in molecular crystals. Understanding the interplay between the luminophore and its environment is crucial for designing more efficient materials from first principles. 

To this end, this work approaches the problem using computational chemistry methods to investigate organic compounds exhibiting \ac{AIE}. AIE-active compounds are non-emissive in dispersed media, but undergo a switch-on of luminescence, typically in the form of fluorescence, upon aggregation. Since technological applications require a thin-film or solid-state layer, AIE has attracted considerable interest as a pathway to overcome the common effect of \ac{ACQ}, hitherto a major obstacle in the development of organic luminophores. The potency of \ac{AIE} systems can be further boosted by incorporating \acf{ESIPT} into the chromophore, where the large Stokes shift separates emission from absorption and emission and increases the quantum yield of fluorescence. ESIPT systems exhibiting AIE are the focus of this thesis. 

In this chapter the problem of \ac{ACQ} shall be introduced, where we consider the photophysics of aromatic molecular aggregates. Following this is an examination of the \ac{AIE} phenomenon through analysis of typical structures and mechanistic interpretations, before introducing the \ac{ESIPT} mechanism. Once the key concepts of \ac{ACQ}, \ac{AIE}, and \ac{ESIPT} have been established, we focus on the systems under investigation - two families of related compounds with a range of photochemical behaviours. The \acp{HC} have a donor-acceptor structure and undergo \ac{ESIPT}, where substituent identities and molecular packing determine the AIE response. In contrast, the four molecules of the \ac{HP} family combine \ac{ESIPT} with a remarkable \ac{AIE} response. At the end of Chapter \ref{chapter: luminscent organic materials} we survey the experimental studies on these systems and set the aims and objectives of this project. 

In Chapter \ref{chapter:theory}, the fundamental principles of quantum chemistry are introduced, with an overview of the computational methods which have been developed in attempt to solve the \schro{} equation. The key concepts of photochemistry are discussed with a focus on excited states in molecular crystals.

In Chapter \ref{chapter:NRdecay}, we present a study of the vacuum photochemistry of five \ac{HC} compounds synthesised by Cheng and coworkers (Figure \ref{figure: HC_experimental}). Here the vertical excitations, \acp{PES} and nonadiabatic dynamics of the family are studied to determine the nonradiative decay pathways and rationalise the nonemission of these compounds.

In Chapter \ref{chapter: Inter} the focus moves to the crystalline photochemistry for the \acp{HC}. Using a hierarchy of QM:MM models, we construct the \ac{PES} and the decay pathway in the excited state for two systems with differing emission behaviour using density functional and multireference methods. The methods employed consider exciton coupling, long and short range electrostatics, and cluster size, to decouple the different factors responsible for the observed photobehaviour. With our results, we establish design rules for efficient ESIPT emitters.

In Chapter \ref{chapter: Connecting}, the scope of our work is expanded to consider the \ac{HP} family, as well as two additional \textbf{HC} compounds which undergo AIE (Table \ref{table: chalcones}. In total we consider eleven compounds to understand the remarkable efficiency of the \ac{HP} family and to test the design rules established in Chapter \ref{chapter: Inter}. A quantitative description of the crystal morphologies is developed, based upon the dimer configurations present. The thesis concludes with an overview of the main results, and how these precipitate the design rules for more efficient organic luminescent materials. 
%%%%%%%%%%%%%%%%%%%%%%%%%%%%%%
\section{Photophysics of Molecular Aggregates}\label{section: lom ACQ}
%%%%%%%%%%%%%%%%%%%%%%%%%%%%%%
\subsection{Molecular Stacking}
Photoinduced luminescence occurs after chromophores with $\pi$-conjugation, typically in the form of aromatic moieties, absorb light in the \ac{UV} or visible region of the electromagnetic spectrum. In good solvents or dilute concentration, fluorescence will usually follow, although the presence of metals or second row elements can allow phosphorescence \textit{via} intersystem crossing. Upon aggregation with increased concentration, or crystallisation, the fluorescence is often reduced or quenched completely. 

In 1954, photochemistry pioneer Theodor F\"{o}rster elegantly showed that the fluorescence of pyrene is shifted and weakened with increasing concentration (Figure \ref{figure: Forster_Spectra}).\cite{Forster1954,Forster1969} Aromatic rings in conventional luminophores lead to intermolecular $\pi$-$\pi$ interactions and the formation of dimers. In the case of pyrene, as the concentration increases a new fluorescent species is formed and the original emission band loses intensity. Since the absorption spectrum is unchanged in the concentration range, the arising band can be ``attributed to an associate that exists only in the excited electronic state, i.e. to an excimer."\cite{Forster1969} The formation of excimers can be detrimental to the fluorescence of organic systems.
\begin{figure}[t]
\centering
  \includegraphics[width=0.6\linewidth]{1Intro/Forster_Spectra.pdf}
  \caption[Fluorescence spectrum of pyrene]{Fluorescence spectrum of pyrene (in n-heptane, 20\degree{}C) at different concentrations: a) \SI{5.0e-5}, b) \SI{1.8e-4}, c) \SI{3.1e-4}, d) \SI{7.0e-4}{mol L^{-1}}. Reprinted from ref.~\citenum{Forster1969} with permission of Wiley-VCH.}
  \label{figure: Forster_Spectra}
\end{figure}

The supermolecular alignment of aromatic groups is commonly termed $\pi$-stacking or $\pi$-$\pi$ interactions in the chemical literature. However, this labelling can be slightly misleading and even inaccurate.\cite{Grimme2008,Martinez2012} It  is argued by Grimme that a specific $\pi$-$\pi$ interaction arises only in large, polyaromatic groups as a result of increased dispersion in specific orientations.\cite{Grimme2008} Meanwhile, a thorough review of experimental and theoretical literature by Martinez and Iverson found a lack of the face-centred stacking of aromatic groups which would maximise overlap of aromatic $\pi$ clouds.\cite{Martinez2012} They argue that the terms $\pi$-stacking and $\pi$-$\pi$ interactions are misnomers since they incorrectly imply the ubiquity of face-face stacking configurations. Typical stacking arrangements are shown in Figure \ref{figure: Benzene_Stacking}, where parallel displacement tempers the unfavourable electrostatic interaction and reduces the Pauli exchange repulsion, with the dominating factor being the favourable dispersion interaction. Inclusion of substituents introduces a permanent dipole, with substituents preferentially aligning antiparallel.\cite{Martinez2012}

\begin{figure}[t]
\centering
  \includegraphics[width=0.7\linewidth]{1Intro/Stacking.pdf}
  \caption[Stacking arrangements of benzene]{Packing arrangements of benzene molecules: a) face-centred, b) displaced, c) Perpendicular T-shaped, d) Perpendicular Y-shaped. Adapted from ref.~\citenum{Martinez2012}.}
  \label{figure: Benzene_Stacking}
\end{figure}

This intermolecular interaction is detrimental to solid-state fluorescence, and yet is a direct consequence of the design requirements of the chromophore. Researchers have developed numerous strategies to overcome this effect. In biosensing applications, work is often carried out in dilute solutions with reduced sensitivity because of \ac{ACQ}.\cite{Thomas2007,Kwok2015} In the solid state, for instance in thin films for optoelectronics, the \ac{ACQ} effect means that solution screening for viable candidates is rendered meaningless by the differing luminescent properties of the final material compared to the molecule. Strategies to circumvent \ac{ACQ} have seen varying success, for example through the inclusion of bulky substituents, polar groups, and promotion of hydrogen bonding.\cite{Hong2009,Zhang2013,Mei2014,Mei2015} However, synthetic modification can in turn affect the chromophore's electronic structure and its excited state properties, thus commencing a tedious trail-and-error optimisation process. Attempts to physically block aggregation by encapsulating in surfactants or polymer matrices require extensive engineering and can reduce charge transport.\cite{Hong2009,Chen2000,Lee2013} 

The deleterious effects of \ac{ACQ} cannot be underestimated and pose a significant problem for organic luminescent applications. In the next section, we shall look in more detail of the photophysics of molecular aggregates, within the context of Kasha's exciton theory.
%%%%%%%%%%%%%%%%%%%%%%%%%%%%%%
\subsection{Exciton Theory}\label{section: lom intermolecular-interactions}
Intermolecular interactions become photophysically important in the solid state due to the dense packing of molecular units. This effect is typically framed in Kasha's two state exciton theory.\cite{Gierschner2009,Gierschner2013,Gierschner2013a,Hestand2017,Shi2017} The exciton is typically defined as a delocalised excited state, where the excited electron and hole remain in close proximity. The Coulomb interaction between the transition dipoles of the monomers in a dimer results in an energy shift in the absorption spectrum, and has underpinned exciton theory since its inception in the 1960s.\cite{Kasha1965a} When two monomers stack ``side-by-side", the Coulombic excitonic coupling is positive, a blue-shifted absorption spectrum is witnessed (relative to the isolated monomer) and radiative decay is reduced. This stacking arrangement is denoted a H-aggregate. Conversely, in J-aggregates, dimers align ``head-to-tail", and a red-shifted absorption is witnessed with an increased radiative decay. These two extreme cases are depicted in Figure \ref{figure: H_J_Aggregates}, and have helped interpret the supermolecular photophenomena in molecular aggregates.

When the intermolecular distance is large enough to minimise
orbital overlap, the exciton can be considered as the interaction of the wavefunction of one monomer ($\ket{1}$) with the wavefunction of the second monomer ($\ket{2}$). In the excited state, the  wavefunctions interact to form a delocalised exciton (a Frenkel exciton), the Hamiltonian $\hat{H}$ of which is given
\begin{equation}
\hat{H}=\hbar\omega + \mathrm{\hbar{}D} + J_{12}\{\ket{1}\bra{2}+\ket{2}\bra{1}\}.
\end{equation}
The first term is the energy gap between S\textsubscript{0} and S\textsubscript{1},  $\hbar{}D$ is the shift of the S\textsubscript{1} state in going from the vacuum to the crystal, and $J_{12}$ is the Coulombic coupling.\cite{Spano} The exciton for a multisite system can be described using periodic boundary conditions to produce a wavelike function $k$. For the two site system, $\ket{k}$ is\cite{Hestand2018}
\begin{equation}
\ket{k}=\frac{1}{\sqrt{2}}\big[e^{ik}\ket{1}+  e^{i2k}\ket{2}\big]\qquad\quad k=0,\pm{\pi}.
\end{equation}
For the allowed values of $k$, the transition energy $E_k$ for state $k$ is an eigenvalue of $\hat{H}$, and has the form
\begin{equation}
\mathrm{E}_{k}=\hbar\omega + \mathrm{\hbar{}D} +J_{12}\cos{k}.
\end{equation}
For the two site system, this results in two exciton states in the Frenkel Hamiltonian, as depicted in Figure \ref{figure: H_J_Aggregates}. The Coulomb coupling $J$ represents the interaction energy due to exchange of excitation energy between $\ket{1}$ and $\ket{2}$, the sign of which dictates whether the symmetric superposition is the upper or lower state. When the coupling is positive, the symmetric state is the upper state, as in the H-aggregate, and the transition dipoles are reinforced resulting in blue-shifted absorption compared to the gas-phase monomer. Since Kasha's rule dictates that emission will usually take place from the lower state, the state is nonradiative and fluorescence is quenched.\cite{Kasha1950} When the coupling is negative, the symmetric state is lowered and emission is symmetry allowed.\cite{Hestand2017}
\begin{figure}[t]
\centering
  \includegraphics[width=\linewidth]{1Intro/H_J_aggregates.pdf}
  \caption[Exciton energy level diagram for H- and J-aggregates]{Energy level diagram for a H-aggregate (left) and J-aggregate (right) of naphthalene. For the H-aggregated, the Coulombic coupling $J$ is positive, raising the energy of the symmetric state resulting in blue-shifted absorption with respect to the monomer state. Emission from the lower state is forbidden, and thus quenched. In the J-aggregate, the negative Coulombic coupling results in red-shifted absorption and allowed emission. The bright excitonic state, the symmetric superposition of the monomer wavefunctions, is indicated with a star.}
  \label{figure: H_J_Aggregates}
\end{figure}
\begin{comment}
\begin{equation}
\ket{g_{1}g_{2}}
\end{equation}
\begin{equation}
\ket{e_{1}g_{2}}
\end{equation}
\begin{equation}
\ket{g_{1}e_{2}}
\end{equation}
\begin{equation}
J_{ij}>0
\end{equation}
\begin{equation}
J_{ij}<0
\end{equation}
\begin{equation}
2|J_{ij}|
\end{equation}
\begin{equation}
\frac{\ket{1}+\ket{2}}{\sqrt{2}}
\end{equation}
\begin{equation}
\frac{\ket{1}-\ket{2}}{\sqrt{2}}
\end{equation}
\begin{equation}
\ket{1}
\end{equation}
\begin{equation}
\ket{2}
\end{equation}
\end{comment}
H- and J-aggregates represent the two extreme stacking cases. 

Understanding the complex photophenomena of molecular aggregates is important in the development of materials requiring control over the exciton. The existence of excitons and the dynamics in the solid state opens nonradiative decay channels and often a quenching of fluorescence for typical aromatic systems due to the prominence of H-aggregates. This has hindered the development of solid-state organic luminescence. 

%%%%%%%%%%%%%%%%%%%%%%%%%%%%%%
%%%%%%%%%%%%%%%%%%%%%%%%%%%%%%
\subsection{Methods to Calculate Exciton Couplings}\label{section: lom exciton_couplings}
%%%%%%%%%%%%%%%%%%%%%%%%%%%%%%
Quantum chemical methods allow for the calculation of the coupling parameter  $J$ for two monomers, in general labelled  $i$ and $j$, in a molecular dimer. In Kasha's exciton theory, the coupling $J_{ij}$ arises from the interaction of the transition dipole moments, which at its most crude can be treated as the interaction between two point dipoles,
\begin{equation}\label{equation: PDA}
    J_{ij}=\frac{\mu^{2}(1-3\cos^{2}\theta)}{4\pi\epsilon{}R^3}
\end{equation}
where $R$ is the intermolecular distance (between the centroids of the monomers), $\theta$ is the angle between  $\bm{\mu}$ (transition dipole vector) and $\bm{R}$, and $\epsilon$ is the optical dielectric constant of the medium. Thus for H-aggregates, when $\theta$=90\degree{}, the coupling is positive, and is negative in J-aggregates when $\theta$ is close to zero. H-aggregates become J-aggregates at the so-called ``magic angle" of 54.7\textdegree.\cite{Hestand2018}

The crude point-dipole, Coulomb interpretation of the coupling is only valid approximation for homodimers with large intermolecular separation.\cite{Kistler2013} When monomers stack in close proximity ($R\leq4\si{\angstrom}$), wavefunction overlap invokes the possibility of \ac{CT} states, where the electron resides on one site and the hole on another.\cite{Darghouth2018} This creates a short-range coupling factor as well as the long-range Coulomb coupling. The total coupling is the combined short-range \ac{CT} coupling and the Coulomb coupling, leading to complex behaviour which is dependent on the sign of the short- and long-range contributions. For example, the \ac{CT} coupling is highly sensitive to the molecular geometry, where small distortions can change the total coupling value and result in multiple H- to J-aggregate conversions.\cite{Arago2015} Yamagata \textit{et. al} devised a wavefunction overlap scheme to calculate the effect of charge transfer in addition to  the Coulomb interpretation.\cite{Yamagata2012} %The two components of the total coupling has lead to an expanded nomenclature for stacking motifs, where the contribution of both the short and long-range coupling is considered (HH, HJ, JH, JJ). In the JH and HJ, the interference is completely destructive and the total coupling is zero, resulting in an absorption spectrum at the same position as a monomer.\cite{Hestand2017}

Perhaps the simplest method to attain $J_{ij}$ is to refer to the original definition of the coupling shown in Figure \ref{figure: H_J_Aggregates} and use the splitting of the first two adiabatic excited states of the supermolecular system,
\begin{equation}
    J_{ij}=\frac{E(S_{2})-E(S_{1})}{2}
\end{equation}
It is a method which has proven popular for its conceptual simplicity and ease of calculation, since only one calculation on the dimer is required.\cite{Hsu2009,Gierschner2013a,Shi2017}

An alternative method based on the Coulomb interpretation is to project the full transition density matrix as atomic transition charges charges $q^{t}$, where each atom in the monomer is assigned a transition charge. This is cheaper than the full supermolecular calculation since only a monomer need be calculated, although the polarising effect of the second monomer is therefore neglected. In such a scheme, $J$ is calculated through intermolecular pairwise contributions
\begin{equation}\label{equation: J_TC}
\sum_{u}^{N_{i}}\sum_{v}^{N_{j}}\frac{q^{t}_{u}q^{t}_{v}}{|\bm{R}_{u}^{i}-\bm{R}_{v}^{j}|}
\end{equation}
between atom $u$ monomer $i$ and atom $v$ on monomer $j$. This method is capable of reproducing the supermolecular couplings at lower computational costs for well-separated dimers.\cite{Kistler2013} Alternatively, the transition densities can be projected onto a three-dimensional real-space grid and integrated to give a molecular transition density, the product of which gives the coupling, in what is known as the transition density cube method.\cite{Bricker2014}

An attractive method to calculate $J_{ij}$, incorporating both the Coulomb and short-range interactions, is the diabatization method devised by Arag\'o and Troisi. Here, the adiabatic energies of the supermolecule in the diagonal adiabatic Hamiltonian are transformed to the diabatic, monomer basis, such that the off-diagonal elements of the diabatic Hamiltonian \textbf{H}\textsuperscript{\textit{D}} contain the couplings, computed \textit{via}
\begin{equation}
\textbf{H}^{D}=\textbf{C}\textbf{H}^{A}\textbf{C}^\dag
\label{equation: diabatic hamiltonian}
\end{equation}
\begin{equation}
\label{equation: diabatic matrix}
\textbf{H}^D=
\begin{bmatrix}
E_{i}^D & J_{ij}\\
J_{ij} & E_{j}^D
\end{bmatrix}
=
\begin{bmatrix}
C_{11} & C_{12}\\
C_{21} & C_{22}
\end{bmatrix}
\begin{bmatrix}
E_{i}^A & 0\\
0 & E_{j}^A
\end{bmatrix}
\begin{bmatrix}
C_{11} & C_{21}\\
C_{12} & C_{22}
\end{bmatrix}
\end{equation} 
where \textbf{H}\textsuperscript{\textit{A}} is the diagonal Hamiltonian of the \sone{} and \stwo{} excitation energies and \textbf{C} is the adiabatic-diabatic transformation matrix. There exists no unique way to transform the bases, and various properties can be used, such as atomic transition charges or transition dipoles. In our work, \textbf{C} is defined as the matrix which minimised the difference between the transition dipole moments of the first two excited states of the dimer ($\bm{\mu}^{A}$) and the transition dipole moments of the first excited state of the two isolated monomers ($\bm{\mu}^{ISO}$), which are collected by matrix \textbf{M}. The singular value decomposition method of \textbf{M} into $\bm{U\Sigma{}V}^{\dag}$ yields matrix \textbf{C},
\begin{equation}
\begin{split}
    \bm{\mathrm{M}}&=(\bm{\mu}^{A})^{\dag}\bm{\mu}^{ISO}\\
    &=\bm{U\Sigma{}V}^{\dag}
    \end{split}
\end{equation}
\begin{equation}
    \bm{\mathrm{C}}^{\dag}=\bm{UV}^{\dag}
\end{equation}
This method encapsulates both the Coulomb interaction and the short-range effects such as exchange, overlap, and charge-transfer.\cite{Arago2015,Arago2016,Fornari2017} In this thesis, we implement the method of Troisi and evaluate exciton couplings in Chapter \ref{chapter: Inter}.  In chapter \ref{chapter: Connecting}, we explicitly compare the results to the couplings obtained using the supermolecular approach, and extend the method to a trimer basis, where the effect of a third monomer on the coupling is considered.











%%%%%%%%%%%%%%%%%%%%%%%%%%%%%%
%%%%%%%%%%%%%%%%%%%%%%%%%%%%%%
\section{Aggregation Induced Emission}\label{section: lom AIE}
%%%%%%%%%%%%%%%%%%%%%%%%%%%%%%
\subsection{Overview}
In 2001, the Tang group at the Hong Kong University of Science and Technology were interested in silole-based polymers for highly emissive thin-films. As was common at the time, fabrication of such materials was challenging due to \ac{ACQ}. Through serendipity during a purification process, they found that a wet spot of 1-methyl-1,2,3,4,5-pentaphenylsilole was almost non-emissive, but brightly fluorescent after solvent evaporation.\cite{Luo2001} The law of aggregation quenching emission had been turned on its head, and the Tang group had observed the exact opposite behaviour, of molecular aggregation inducing light emission. 

The Tang group used the \ac{AIE} phenomenon to develop a range of chemical sensors to detect volatile organic compounds, explosives, and pH.\cite{Dong2007,Li2005,Li2009} Such was the magnitude of the \ac{AIE} breakthrough and the mechanistic interpretations provided by the Tang group, many other groups began to explore this exciting new phenomenon for a wide range of applications. In the field of biological probing, \ac{AIE}-active systems can detect important small molecules such as glucose, thiols, and lactic acid.\cite{Wang2014,Yuan2014,Shen2012} Probes have been developed to detect protein fibrillation, which has been linked to Alzheimer's, Parkinson's and type II diabetes.\cite{Hong2012} \ac{AIE} systems have been also used in medical imaging, where fluorescence is an attractive technique due its high resolution, wide applicability, and low cost.\cite{Mei2015} The Tang group have tracked the progress of the field with periodic, in-depth reviews of the vast number of innovations, of which they contribute a significant share.\cite{Hong2009,Wang2010a,Hong2011,Mei2014,Hu2014,Mei2015} The most prominent avenue for the application of AIE is in optoelectronic and lasing devices, particularly OLEDs, which we will overview in Section \ref{section: lom AIE_applications}. Prior to that, in the next section,  we examine the proposed mechanisms which precipitated from the experimental studies led by Tang \textit{et. al}.

\subsection{Hypothesised Mechanisms}\label{section: lom AIE_mechanisms}

\begin{figure}[t]
\centering
  \includegraphics[width=0.7\linewidth]{1Intro/HPS_TPE.pdf}
  \caption[Examples of AIE-active chromophores]{Two dimensional (top) and three dimensional (bottom) structures of two of the ubiquitous AIE-active systems, hexaphenylsilole (HPS, left) and tetraphenylethene (\ac{TPE}, right). AIE occurs through restriction of the rotational motions depicted with arrows.}
  \label{figure: HPS_TPE}
\end{figure}

Systems exhibiting \ac{AIE} are typically based on a propeller architecture, where a central stator is connected to a number of aromatic rotors \textit{via} single bonds, as shown in Figure \ref{figure: HPS_TPE}. In the initial analysis of 1-methyl-1,2,3,4,5-pentaphenylsilole, the absorption spectra showed that after the water fraction in an ethanol-water solvent mixture rose above 60\%, the absorption band increased in intensity and moved to longer wavelengths.\cite{Luo2001} On this basis, the \ac{AIE} activity was attributed to the formation of aggregates which forced the molecules into a more planar conformation, thus increasing the conjugation and the absorption. Enough rotation about the sigma bonds was still possible to prevent complete planarity and therefore limiting the stacking and subsequent fluorescence quenching. 

A few months later, the group showed the \ac{AIE} effect for four more silole systems, followed by an extensive study of ten phenylsiloles.\cite{Tang2001,Chen2003} Crucially, in this later work the crystal structure of the siloles showed that the conformation remains twisted in the solid state.\cite{Chen2003} Large torsional angles exist between the phenyl groups and the central silole moiety, on account of the steric repulsion between the six phenyl groups. The lack of space between the phenyl blades, and their angular orientations, prevents intermolecular stacking which give rise to aggregation caused quenching. Therefore the planarisation hypothesis was wrong. 

The group investigated other quenching mechanisms, such as \ac{TICT}, where a non-emissive twisted conformer is formed \textit{via} charge separation in the chromophore. The \ac{TICT} state can be stabilised, and thus favoured, in polar solvents. However, no emission was recorded across a wide range of solvent polarities, ruling out \ac{TICT} as the cause for \ac{AIE} in the siloles. Also, the lack of electron donor and acceptor groups make it highly unlikely that \ac{TICT} could take place in the siloles. In an elegant experiment it was found that the emission increased almost linearly with increasing solvent viscosity, even though no aggregates were formed. In a similar vain, the photoluminescence intensity increased with decreasing temperature, with NMR studies confirming that the intramolecular rotations of the exterior phenyl groups were reduced at lower temperatures. It was concluded that in good solvents, at ambient temperature, the energy consumed by the rotation of the phenyl groups about the single bonds results in the nonradiative decay of the excited state decay. Upon aggregation, the propeller-like shape limits the intermolecular stacking while C-H\textperiodcentered\textperiodcentered\textperiodcentered$\pi$ interactions rigidify the structure, hindering the intramolecular rotation and the excited state decays \textit{via} fluorescence.\cite{Chen2003}

The \ac{RIR} mechanism allowed the library of \ac{AIE}-active systems to be expanded to other molecules with propeller-shaped structures. Along with the phenylsiloles, \ac{TPE} derivatives (Figure \ref{figure: HPS_TPE}) have driven understanding and technological progress in the field.\cite{Hong2009,Wang2010a,Hong2011,Mei2014,Hu2014,Mei2015} By switching a phenyl group of \ac{TPE} with a traditional \ac{ACQ} molecules, such as triphenylamine or carbazole, the electroluminsecence properties of the system are enhanced due to the hole-transport properties of the \ac{ACQ} group.\cite{Chan2014} Conversely, \ac{ACQ} cores can be made to undergo \ac{AIE} by the attachment of \ac{TPE}.\cite{Yuan2010a}

Dissipation of the excited state can occur through means other than rotation. A bent, $\pi$-surface system of benzenes fused with cyclooctatraenes undergoes \ac{AIE} but without any rotable units.\cite{Nishiuchi2013} In solution, ring inversions dissipate the excited state and there is almost no emission. Single crystals show emission in the blue region, since the bent structure prevents facial stacking and the inversion modes are restricted. Thus \ac{AIE} is achieved through \ac{RIV}. In a similar vain, the \ac{RIV} mechanism can be applied to \ac{TPE} by locking pairs of phenyl groups with eythlene linkers. In 2014, Tang unified the \ac{RIR} and \ac{RIV} interpretations under the \ac{RIM} umbrella.\cite{Leung2014}

%%%%%%%%%%%
%%%%%%%%%%%
\subsection{Optoelectronic Applications of AIE}\label{section: lom AIE_applications}
%%%%%%%%%%%
%%%%%%%%%%%
Upon discovery of \ac{AIE}, the potential for the improvement of optoelectronic devices was immediately apparent. In the original publication, the group built a highly emissive blue-emitting electroluminescent device, and optimised the device to 8\% \ac{EQE} in the cyan region, a vast improvement on the previous high of just 1.5\% for an \ac{OLED}. \cite{Luo2001,Chen2002} \ac{EQE} is the product of the electroluminescence efficiency of the emitter (the organic layer in this case) and the external coupling factor, which is a measure of the fraction of light able to escape the \ac{OLED}. Due to electroluminescence efficiency being limited to 25\% for singlet emitters, and the external coupling being limited to around 22\%, it was previously thought that the theoretical maximum \ac{EQE} is 5.5\%. Indeed, such was the remarkable \ac{EQE} measure in this \ac{OLED} that these previously accepted limitations had to be reconsidered.

In follow-up work, a light-blue emitting \ac{OLED} with hexaphenylsilole (Figure \ref{figure: HPS_TPE}) as the emitting layer was fabricated with \ac{EQE} of 7\%.\cite{Chen2003} While the unfavourable spin statistics inhibit efficiency for singlet-emitting \acp{OLED} across the visible spectrum, deep blue emitters are harder still since the large band gap makes charge injection difficult, hindering the development of full colour displays. Non-doped deep blue \acp{OLED} with \ac{EQE} of around 4\% were reached in 2014, with emitters based on triphenylamine and tetraphenylethene (Figure \ref{figure: HPS_TPE}).\cite{Huang2014,Huang2014a} This has recently been increased to 6.5\% using a carbazole-based organic layer, and in 2018 reached 9.4\%.\cite{KumarKonidena2017,Tang2018} Inclusion of phosphorescent or thermally-activated delayed fluorescent dopants can further increase the quantum efficiency by harvesting triplet states for luminescence.\cite{Zhu2018}

%%%%%%%%%%%%%%%%%%%%%%%%%%%%%%
\section{Theoretical Framework}\label{section: lom theory}
%%%%%%%%%%%%%%%%%%%%%%%%%%%%%%
%%%%%%%%%%%%%%%%%%%%%%%%%%%%%%
\subsection{Approaches Based on Fermi Golden Rule}\label{section: lom FGR}
%%%%%%%%%%%%%%%%%%%%%%%%%%%%%%
In 2005, the first theoretical study to determine the root of AIE in the phenylsiloles was presented.\cite{Yui2005} To probe the origins of the AIE phenomenon, Shuai \textit{et al.} constructed a model for the observed \ac{QE}, 
\begin{equation}\label{equation: QE}
    \Phi_{\mathrm{PL}}=\frac{k_{r}}{k_{r}+k_{nr}}
\end{equation}
where the overall observed \ac{QE} is determined by the rates of radiative ($k_{r}$) and nonradiative ($k_{nr}$) decay. $k_{nr}$ will typically consist of internal conversion, intersystem crossing, and intermolecular processes such as charge transfer. The idea is to calculate the radiative and nonradiative rates in solution and aggregate form to elucidate the reason for the increased fluorescence in the solid state. This strategy underpins most of the theoretical investigation into AIE, of which the Shuai group have been at the forefront over the last 10 years. 

To evaluate $k_{r}$, the Einstein spontaneous emission formula can be applied,\cite{Yui2005}
\begin{equation}\label{equation: Einstein_rate}
    k_{r}=\frac{fE_{if}^{2}}{1.499}
\end{equation}
where $f_{if}$ is the oscillator strength and $E_{if}$ is the energy difference between the initial (\sone{}) and final electronic state (\szero{}). This considers that there is no displacement between the ground and excited state \acp{PES}. An alternative method to evaluate $k_{r}$ is to simulate the fluorescence spectrum using a Winger ensemble and integrate the spontaneous emission for the rate,
\begin{equation}\label{equation: integrated_emission_rate}
    k_{r}=\frac{1}{\hbar}\int{\Gamma_{r}(E)dE}
\end{equation}
where $\Gamma_{r}$ is the rate of spontaneous emission per molecule per unit of angular frequency between $E/\hbar$ and $(E+d
E)/\hbar$.\cite{Niu2010,Crespo-Otero2012}
We compare both strategies in Chapter \ref{chapter: Connecting}.  

The Fermi Golden rule and Condon approximation can be used within the Born-Oppenheimer approximation to determine $k_{nr}$ according to,\cite{Yin2006}
\begin{equation}\label{equation: Fermi}
    k_{nr}(T)=\frac{2\pi}{\hbar}\sum_{\nu_{i}\nu_{f}}P_{i\nu_{i}}(T)\bigg|\sum_{k}\bra{\Phi_{f}}\hat{P}_{k}\ket{\Phi_{i}}\bra{\Theta_{f\nu_{f}}}\hat{P}_{k}\ket{\Theta_{i\nu_{i}}}\bigg|^{2}\delta{}(E_{i\nu_{i}}-E_{f\nu_{f}})
\end{equation}
where $P_{i\nu_{i}}$ is the Boltzmann distribution of vibrational states ($\nu$) in the initial electronic state ($\Phi_{i}$), $\hat{P}$ is the momentum operator to give the coupling between electronic states and nuclear ($\Theta$) wavefunctions. The first sum runs over each of the vibrational states in the initial ($\nu_{i}$) and final ($\nu_{f}$)  states, and Boltzmann probability for the excited state (($\nu_{i}$) is multiplied by the product of the electronic (or nonadiabatic) coupling and the vibrational coupling. The coupling terms measure the change of the nuclear and electronic wavefunctions due to nuclear displacements for each nuclei $k$. For systems with small spin-orbit coupling, the contributions of intersystem crossing are normally neglected. As such, the framework for evaluating $k_{nr}$ focuses on internal conversion processes, although intermolecular decay through exciton formation or charge transfer are of growing interest.\cite{Li2017} Equation \ref{equation: Fermi} lies at the heart of Shuai's theoretical framework for interpreting AIE through intramolecular motions, which we denote the \ac{FGR-RIM} approach.

Solving Equation \ref{equation: Fermi} is computationally demanding and thus approximations must be made. Initially Shuai \textit{et al.} used a displaced harmonic approximation, where it is assumed that the excited state \ac{PES} is just a rigid displacement of the ground state.\cite{Yui2005} The internal conversion rate can then be recast as
\begin{equation} \label{equation: displaced_harmonic}
    k_{nr}=\sum_{l}\frac{1}{\hbar}\bigg[\frac{\omega_{l}}{2\hbar{}}\big|R_{fi}^{l}\big|^{2}\bigg]N_{FC}
\end{equation}
where $R^{l}_{fi}$ are the nonadiabatic couplings, $\omega_{l}$ is the frequency of mode $l$ and $N_{FC}$ is the density-weighted Franck-Condon factor, given by
\begin{equation}\label{equation: NFC}
N_{FC}=\sqrt{\frac{2\pi}{\sum_{j}S_{j}\omega^{2}_{j}(2\bar{n}_{j}+1)}}\mathrm{exp}\bigg[-\frac{(\omega_{fi}+\omega_{l}+\sum_{j}S_{j}\omega_{j})^{2}}{2\sum_{j}S_{j}\omega_{j}^{2}(2\bar{n}_{j}+1)}\bigg].
\end{equation}
$k_{nr}$ the weighted Franck-Condon factors are functions of the \ac{HR} factors $S$ for each mode $j$, which determine the difference in vibrational wavefunctions in different electronic states. 

Within the displaced harmonic oscillator model (Equation \ref{equation: displaced_harmonic}), some key approximations are made. Firstly, by assuming the ground and excited state \acp{PES} mirror each other means that the model is only valid for systems where the excited state geometry does not deviate far from the ground state equilibrium position. Secondly, in the implementation a ''promoting-mode" $l$ is chosen which is used to calculate the nonadiabatic coupling but does not contribute to $N_{FC}$ (Equation \ref{equation: NFC}). 

To overcome these limitations, Shuai \textit{et al.} further developed the \ac{FGR-RIM} theory to incorporate the \ac{DRE}, such that all vibrations are considered rather than just one promoting mode.\cite{Yui2005,Peng2007a,Niu2010} The \ac{DRE} correlates the excited state coordinates $Q^{e}$ with those of the ground state $Q^{g}$
\begin{equation}
    Q^{e}_{i}=\sum_{j}R_{ij}Q^{g}_{j}+D_{i}
\end{equation}
where $R_{ij}$ are the elements of the Duschinsky rotation matrix and $D_{i}$ is the displacement of the two \ac{PES}. Shuai \textit{et al.} solve Equation \ref{equation: Fermi} with \ac{DRE} through a path-integral formalism.\cite{Peng2007a} Fourier transform of the delta function yields the thermal vibrational correlation function which can be solved analytically \textit{via} the implementation of Peng and Shuai.\cite{Peng2013}
%\begin{equation}
%k_{nr}=\sum_{kl}\frac{1}{\hbar}R_{kl}\int_{\infty}^{\infty}\%diff{t}\big[e^{i\omega_{if}t}Z_{i}^{-1}\rho_{IC}(t,T)\big]
%\end{equation}
Incorporating \ac{DRE} to model the effect of temperature shows that increases $k_{nr}$ by 700 times when the temperature increases from 70K to 300K.\cite{Peng2007} Low-frequency modes are strongly coupled for tetraphenyl butadiene compounds, while high-frequency modes are less sensitive to temperature effects and less affected by \ac{DRE}. 

Shuai \textit{et al.} have also considered the effect of including exciton couplings in the nonradiative decay rate.\cite{Li2017} The authors developed an analytical expression for $k_{nr}$ incorporating the exciton coupling and evaluated the rates for four aromatic molecules and four typical AIE chromophores. It was found that the the exciton coupling increases $k_{nr}$, but only slightly. However, the values of exciton couplings for the tested systems was less than 30meV, which is relatively small.

The \ac{FGR-RIM} framework offers major progress in rationalising AIE through the restriction of intramolecular motions interpretation. Shuai's group have implemented the methods in a closed-source, commercial code MOMAP.\cite{Niu2018} However, the formalism contains some key approximations which should be highlighted. For the electronic nonadiabatic coupling, perturbation theory is used where the magnitude of the coupling is only calculated at the equilibrium geometries in \szero{} and \sone{}. This may be a source of error in systems which deviate from the equilibrium structures, as nonadiabatic couplings can vary significantly in the regions of surface crossings. Further, low-frequency modes driving the photochemistry can be highly anharmonic and nonadiabatic couplings in the equilibrium region do not necessarily correlate with the deactivation modes driving the photochemistry. In general, the use of the harmonic potentials potentially limits the applications to molecules which excited state dynamics is restricted to regions of the PES around equilibrium. These approximations can break down in systems with more highly anharmonic excited state \acp{PES}. 
%%%%%%%%%%%%%%%
%%%%%%%%%%%%%%%
\subsection{Huang-Rhys Factors and Reorganisation Energies}\label{section: lom HR}
%%%%%%%%%%%%%%%
%%%%%%%%%%%%%%%
Using the Duschinsky rotation matrix, normal modes can be correlated between different electronic states, yielding the Huang-Rhys factors $S$ between the electronic states for each frequency $\omega$
\begin{equation}
S_{j}=\frac{\omega_{j}D_{j}^2}{2\hbar{}}
\end{equation}
where $D_{j}$ is the displacement of mode $j$ between the equilibrium geometry in the considered electronic states. Summation of each Huang-Rhys factor yields the normal mode reorganisation energy for the system,
\begin{equation}
\lambda_{NM}=\sum_{j=1}^{3N-6}\hbar{}\omega_{j}S_{j}
\end{equation}
which is associated with the energetic cost of geometry relaxation in the electronic transition. Analysis of the Huang-Rhys factors, the reorganisation energies, and the nonadiabatic couplings give insight into the main vibrational modes involved in the internal conversion. In particular, analysis of the HR factors and reorganisation energies are popular for the semi-quantitative interpretation of AIE.\cite{Yin2006,Peng2007,Li2011,Peng2013,Shuai2014c,Shuai2014,Wu2014,Zhang2015a,Zheng2016,Fan2016,Zhang2016,Duan2017,Fan2017,Fan2018} Since $k_{nr}$ is a product of the electronic couplings and the Franck-Condon factors, both of these must be small in order to increase \ac{QE} (Equation \ref{equation: QE}). For the phenylsilole compounds, calculation of $k_{r}$ and $k_{nr}$ in vacuum revealed that the most significant contributions to $k_{nr}$ come from low-frequency rotational modes, which have large \ac{HR} factors and large reorganisation energies. The authors hypothesise that in the solid state, the dampening of rotational modes due to steric hindrance reduces the \ac{HR} factor, and thus the $N_{FC}$ contribution to $k_{nr}$.\cite{Yui2005,Yin2006} $k_{nr}$ decreases with respect to $k_{r}$, resulting in fluorescence.

The reorganisation energy can also be calculated in the adiabatic regime,\cite{Reimers2001}
\begin{equation}
\begin{split}
\lambda_{A}&=\lambda_{ex}+\lambda_{g}\\
&=(E_{ex}^*-E_{ex})+(E_{g}^*-E_g)
\label{equation: lambda}
\end{split}
\end{equation}
where $E_g$ and $E_{ex}$ are ground state energies at the \szero{} and \sone{} minima, and $E_{g}^*$ and $E_{ex}^*$ are the corresponding excited state energies at respective minima.\cite{Bredas2004} This is visualised in Figure \ref{figure: reorganisation_energy}. $\lambda_{A}$ can be easily calculated using quantum chemical methods, where geometry optimisation in \szero{} and \sone{}, and a single point calculation of \sone{} at the \szero{} minimum, yields four quantities required to evaluate Equation \ref{equation: lambda}. This does not require the normal modes to be calculated, which can be computationally expensive in the excited state for large molecules.

\begin{figure}[t]
\centering
  \includegraphics[width=0.6\linewidth]{1Intro/reorganisation_energy.pdf}
  \caption[Schematic of the reorganisation energy in the adiabatic regime]{Schematic of how the reorganisation energy $\lambda$ is calculated in the adiabatic regime, based on the ground and excited state \acp{PES} of a monomer and the four quantities of Equation \ref{equation: lambda}. Red arrows indicate vertical transitions between electronic states. Adapted from reference \citenum{Stehr2014}.}
  \label{figure: reorganisation_energy}
\end{figure}
%%%%%%%%%%%%%%%
%%%%%%%%%%%%%%%
\subsection{Restricted Access to Conical Intersection Model}\label{section: lom RACI}
%%%%%%%%%%%%%%%
%%%%%%%%%%%%%%%
A different approach to computational modelling of AIE is to consider the topology of the whole \ac{PES} across the photochemical reaction coordinate. While the framework adopted by Shuai considers the vibronic coupling around the equilibrium geometries, insight into the AIE mechanism can be found by considering the role of conical intersections. Conical intersections are points of degeneracy between electronic states, and allow for nonadiabatic radiationless decay between an upper and lower surface. A more detailed overview on conical intersections is given in Section \ref{section: Photo}. In the \ac{RACI} model of Blancafort \textit{et al.}, an energetically accessible conical intersection is responsible for the nonemission in solution. Upon aggregation, steric constraints result in the conical intersection being too high in energy to be populated, leading to fluorescence.\cite{Peng2016} The \ac{RACI} is advantageous in that it does not impose restrictions or assumptions of the topology of the \ac{PES}.  

Blancafort and co-workers first identified the role of a conical intersection for systems based on diphenyldibenzofulvene using multiconfigurational methods.\cite{Li2013} In solution, torsional rotation takes the system to an extended crossing seam, with the \ac{MECI} lying 0.7 eV below the excitation energy. In the crystal, rotation is hindered and the \ac{MECI} lies above the excitation energy. In the RIM model, the phenyl substituents dissipate the excited state energy. In the RACI interpretation, the role of the phenyl substituents is to restrict the torsion in the solid state. Wang \textit{et al.} found a similar mechanism for 4-diethylamino-2 benzylidene malonic acid dimethyl ester. \cite{Wang2016} By combining TDDFT, CASSCF, and CASPT2 methods, the authors compared the \acp{PES} in solution and in the solid state. In solution, relaxation along a torsional mode takes the system to a conical intersection which is inaccessible in the crystal. A schematic of the general \ac{RACI} mechanism is depicted in Figure \ref{figure: RACI}.

\begin{figure}[t]
\centering
  \includegraphics[width=0.9\linewidth]{1Intro/RACI_Model.pdf}
  \caption[Schematic of the RACI Model]{Schematic of the RACI model, where in solution there is no emission due to the energetically accessible conical intersection. In crystalline or aggregated state, the minimal energy conical intersection (MECI) is high in energy and inaccessible, which results in AIE.}
  \label{figure: RACI}
\end{figure}

The RACI model has also been applied to explain the different emissive responses of dicyano-distyrilbenzene derivatives with differing CN substitution patterns.\cite{Shi2017} For some derivatives, there is emission in both solution and the solid state, whereas others display AIE, due to the energetic accessibility of a CI in solution. In this case, it is the vertical excitation energy which is altered most between the compounds, with a red-shift making the conical intersection inaccessible. Again, the CI is reached through intramolecular rotation.

Studies have shown that modes other than rotation can be followed to access conical intersections. For \ac{TPE}, dynamics calculations at TDDFT level show that cyclisation through bond formation can lead to nonradiative decay through a CI.\cite{Prlj2016} The formation of photoclyclisation products was later confirmed by transient absorption spectroscopy, while the methylated derivative of \ac{TPE} shows similar behaviour.\cite{Cai2018a,Gao2017} In Duan's study, it was proposed that nonradiative decay of a TPE derivative in solution could be on account of both the intermolecular motions and the conical intersection.\cite{Duan2017} 

Blancafort has shown that the RACI model can account for AIE in Tang's original compounds, the phenylsiloles.\cite{Peng2016} By combining TDDFT and CASSCF/CASPT2 calculations, it is found that a combination of ring puckering and a flapping mode result in a conical intersection in solution, which is inaccessible in the crystal. Conical intersections accessed through ring puckering have been found for other systems exhibiting AIE.\cite{Sasaki2016} Notable is the octatrene derivative, where the AIE character has been attributed to the \ac{RIV} mechanism, as discussed in Section \ref{section: lom AIE_mechanisms}. \cite{Nishiuchi2013} The ring puckering \ac{MECI} uncovered by Yuan \textit{et al.} means it falls under both the \ac{RIM} interpretation and the \ac{RACI} model.\cite{Yuan2013}

The \ac{RACI} model and the \ac{FGR-RIM} interpretation of AIE are different in their underlying assumptions but are not necessarily conflicting interpretations of the AIE phenomenon. The \ac{FGR-RIM} approach determines the nonradiative decay rate from the coupling of electronic and nuclear wavefunctions through nuclear vibrations. The nuclear modes are treated within the harmonic approximation while the electronic coupling is calculated at the equilibrium geometries. This can be troublesome for systems which undergo large nuclear deformation in the excited state, where the modes can become highly anharmonic and the electronic coupling can vary in the vicinity of a conical intersection. In these cases, more accurate lifetimes would be predicted using nonadiabatic dynamics simulations and the \ac{RACI} model can be used to interpret the intersection seam. However, the \ac{RACI} model does require a reliable way to treat all parts of the \ac{PES}, which is not trivial. In particular, accurate excitation energies and conical intersections are required in order to determine the feasible relaxation pathways. For \ac{FGR-RIM}, in the solid state the close packing of molecules reduces the conformational phase space and thus the harmonic approximation is more applicable. The \ac{RACI} model can identify the decay path in solution and the solid state, whilst \ac{FGR-RIM} can provide a quantitative prediction of lifetimes and \ac{QE}.

The discovery, development, and rationalisation of the \ac{AIE} phenomenon has transformed the field of luminescent organic materials. While much of the innovation has been based on the propeller systems, a class of systems based on \ac{ESIPT} have attracted interest in recent years for their favourable photochemical properties. \ac{ESIPT} systems form the basis of the research in this thesis and in the next section the \ac{ESIPT} mechanism shall be introduced, along with the key applications which incorporate \ac{ESIPT}.
%%%%%%%%%%%%%%%
%%%%%%%%%%%%%%%
%\subsection{Charge Transport and Exciton Diffusion}\label{section: lom charge_transport}
%%%%%%%%%%%%%%%
%%%%%%%%%%%%%%%

%%%%%%%%%%%%%%%%%%%%%%%%%%%%%%
\section{Excited State Intramolecular Proton Transfer}\label{section: lom ESIPT}
%%%%%%%%%%%%%%%%%%%%%%%%%%%%%%
\subsection{Combining AIE with ESIPT}
Tautomerism is a type of isomerism resulting in the transfer of a chemical group between two sites on a molecule, and the simultaneous switching of a single and double bond. Photo-induced tautomerism, where the transferring group is a proton, is called \acf{ESIPT}. Research into the mechanism and potential applications of ESIPT has been active for more than half a century, since \ac{ESIPT} was first observed in the 1950s in salicylic acid.\cite{Weller1955} Photochromic materials harnessing \ac{ESIPT} have garnered much attention due to the wide range of applications and remarkable properties. In particular, it is the unusually large Stokes-shifted emission which makes \ac{ESIPT} so attractive. The separation between the absorption and emission bands can typically exceed 200 nm, reducing self-absorption and increasing the output signal for the desired application. The emission colour can be tuned by the addition of electron donating or withdrawing groups, as well as solvent polarity and viscosity.\cite{Azarias2016,Yushchenko2007} Dual emission from the pre- and post-\ac{ESIPT} forms is also possible. These characteristics, in tandem with \ac{AIE}, have resulted in \ac{ESIPT} chromophores being used for chemical sensing, biological imaging and probing, as well the optoelectronic applications such as optical memory, lasers, and \acp{OLED}.\cite{Hsieh2010,Kwon2011,Zhao2012,Demchenko2013,Padalkar2015,Chen2018} However, while there is a huge potential range of applications for ESIPT emitters, there are notable shortcomings, such as low \ac{QE} and short fluorescent lifetimes.\cite{Padalkar2015} 
\begin{figure}[t]
\centering
  \includegraphics[width=0.95\linewidth]{1Intro/ESIPT.pdf}
  \caption[The four-level photocycle of ESIPT]{The four-level photocycle of \ac{ESIPT} for salyclic acid. The HOMO and LUMO orbitals are also shown.}
  \label{figure: ESIPT}
\end{figure}
\subsection{The Four-Level Photocycle}
Inherent in all \ac{ESIPT} processes is a fully reversible four-level photocycle, the prerequisite for which is the presence of an intramolecular hydrogen bond. The proton donor can be an amino or hydroxyl group, while the proton acceptor is usually an imine or carbonyl. The four-level photocycle for salicylic acid is depicted in Figure \ref{figure: ESIPT}, along with the frontier molecular orbitals. The chromophore in the ground state (S\textsubscript{0}) is in the enol form (E), mediating hydrogen bond formation between the carbonyl oxygen and the hydroxyl proton. Upon electronic excitation to \sone{} (\Estar{}), the electronic redistribution acidifies the hydroxyl and increases the basicity of the carbonyl group, as a result of the population of the $\pi^\ast$ orbital on the carbonyl oxygen. In the excited state, the keto form (\Kstar{}) is more stable due to the redistributed electron density, and the proton migrates from the hydroxyl oxygen to the carbonyl oxygen. Depending on the system, fluorescence can occur from both the excited enol form (\Estar) and the excited keto form (\Kstar), although due to the ultrafast nature of the proton transfer the major emitting species is the keto tautomer.\cite{Zhao2012} 

For most \ac{ESIPT} processes containing strong hydrogen bonds, proton transfer is near barrierless and  occurs on a femtosecond time scale.\cite{Padalkar2015} The rate of the proton transfer and emission wavelength are highly sensitive to the surrounding medium and the presence of electron donor/acceptor moieties.\cite{Demchenko2013,Lin2017a,Li2017c} After fluorescence, the ground state keto form (K) is populated and the four level photocycle (E$\rightarrow$E$^{\ast}$$\rightarrow$K$^{\ast}$$\rightarrow$K) is completed. The initial geometry is restored through \ac{GSIPT}, although other photoproducts can be formed, for instance through cis-trans isomerisation or intersystem crossing.\cite{Al-Soufi1990}

\ac{AIE} in \ac{ESIPT} chromophores can be more complex than in non-polar propeller systems. The presence of hydrogen bonding sites enables the formation of intermolecular hydrogen bonds with solvent molecules, weakening the intramolecular bond and hindering \ac{ESIPT}.\cite{Cheng2015f} Kasha showed that the ratio of fluorescence intensity between \Estar and \Kstar dramatically changes based upon the solvent polarity.\cite{Kasha1986} In 3-hydroxyflavanone, the \Kstar fluorescence band is suppressed and the \Estar band increases in intensity with polar solvents. In strongly basic solvents, intermolecular proton transfer can occur between the chromophore and solvent, blocking access to the \Kstar state\cite{Laurent2014}. As touched upon earlier, the presence of donor acceptor groups opens the possibility of deactivation through \acf{TICT}. In solvent, the \ac{TICT} state is populated after proton transfer and lead to nonradiative decay. In a similar vain to \ac{RIR}, aggregation frustrates the torsional mode, preventing the \ac{TICT} state from forming, and opens the radiative decay channel.\cite{Park2017,Wu2015a}

The four-level photocycle in \ac{ESIPT} results in large deviations in the excited state electronic structure, accompanied by large nuclear distortions in solution. As such the application of the \ac{FGR-RIM} interpretation can be potentially problematic. The \ac{RACI} model can be applied to alleviate these difficulties, and many ESIPT systems are known to decay through an accessible \ac{MECI} in solution.\cite{Sobolewski2006,Sobolewski2006a,Barbatti2009,Park2009,Lochbrunner2001,Shigemitsu2012,Sporkel2013,Plasser2014,Baker2015,Park2017,Chen2018} In this work, AIE in the \ac{ESIPT} systems is interpreted mainly through \ac{RACI} model,  with due consideration of the \ac{FGR-RIM} approach in Chapter \ref{chapter: Connecting}.

\subsection{Exploiting ESIPT and AIE for Applications}
The most investigated class of \ac{ESIPT} molecules are based on benzothiazole dyes, particularly \ac{HBT}.\cite{Padalkar2015,Kwon2011} The groups of Li and Liu investigated the effect of solvent for a range of \ac{HBT} derivatives, finding that increasing polarity impedes the proton transfer reaction and diminishes fluorescence. This is compounded by highly polar, protic solvents, where the hydroxyl proton can dissociate to form the phenolic anion. The proton transfer is highly sensitive to the solvent polarity, which is highly useful for sensing and probing applications.\cite{Wang2009,Cheng2015f} A \ac{HBT} analogue has been developed for ratiometric probing for hydrogen peroxide in living cells, where the \ac{ESIPT}-active fluorophore is produced by oxidative hydrolysis.\cite{Tang2018a} \ac{HBT}-based systems are also applicable for pH sensing, ion detection, biothol probing, and intracellular imaging.\cite{Kachwal2018,Kachwal2018,Liu2018}

Substitution of electron donor and acceptor groups onto \ac{ESIPT} cores can alter the proton transfer rate and stability of the enol and keto conformers on the excited state potential energy surface. In an extensive theoretical study, Jacquemin and co-workers investigated how different substitution patterns affect the emission from enol and keto states for a range of benzothiazoles.\cite{Azarias2016} They found that dual emission from both E$^*$ and K$^*$ is only possible in a small energy window for the compounds tested, and that the K$^*$ minimum can be favoured more drastically by dependent on the heteroatom in the core. The strongest substituent effects are seen with electron donor groups, such as methoxy, which stabilise the E$^*$ state. The K$^*$ state can be favoured by electron withdrawing groups in particular positions. Crucially, the effects of combining substituent groups is complex and depend on the specific \ac{ESIPT} core and substituent combination. The easily purturbed electronic structure of these systems make design from first principles extremely challenging.

\ac{ESIPT} cores are excellent candidates for optoelectronic applications on account of minimised self-absorption. However, the environment sensitivity which makes them so suitable for probing can be harmful to device stability.\cite{Kwon2011} Through chemical modifications, the Park group have fabricated stable \ac{OLED} devices with a range of emission frequencies.\cite{Park2008,Park2009,Kim2011} Introduction of carbazole gave access to blue K$^*$ emission, while \ac{EQE} values of 14\% have been reached by incorporating triplet harvesting via thermally activated delayed fluorescence with \ac{ESIPT}.\cite{Park2008,Mamada2017} Emission from both E$^*$ and K$^*$ enables white-light emission in devices, a highly attractive property due to the rarity of single molecule fluorophores with wide emission bands.\cite{Tang2011,Yao2011,Zhang2016b,Serdiuk2017}

The population inversion afforded by the \Kstar form makes \ac{ESIPT} systems attractive for laser applications.\cite{Fang2014,Gierschner2016} Additionally the large Stokes shift limits reabsorption and increases the optical gain.\cite{Kwon2011} In the 1980s, the principle of using \ac{ESIPT} for lasing applications was established by Kasha and co-workers for 3-hydroxyflavone, where the ultrafast proton transfer and double-well excited state potential energy surface lead to efficient population inversion.\cite{Khan1983,Chou1984} Since then, the structural diversity of \ac{ESIPT} systems have produced lasers with emission in the green, orange, and cyan regions.\cite{Sakai2016,Chen2016,Park2012,Park2008} Recently, an imidazole-based system with amplified spontaneous emission properties was developed with deep blue emission.\cite{Park2017} The restriction of the \ac{TICT} state in the crystalline form results in \ac{QE} of 67\%, producing an intense and narrow blue band for emission. 
\subsection{Harnessing ESIPT for Near-IR Emission}
Developing efficient fluorophore emission at the extremes of the visible spectrum is notoriously difficult. Whilst much attention has been paid to the blue region, the red and \ac{NIR} region is also hugely challenging in the solid state. The first system to exhibit solid state lasing properties in the \ac{NIR} region was reported in 2015.\cite{Cheng2015} The compounds were based on \ac{HC} skeletons and displayed \ac{ESIPT}. Interestingly, solid state fluorescence is only witnessed in some analogues, with substituent position and crystal packing modes determining the \ac{QE}. The question of whether electronic effects or the crystal structure determined the fluorescence activity was not fully resolved. Fluorine-containing derivatives with laser properties were published soon after.\cite{Cheng2016} In the same year, the same group published another breakthrough in solid state lasing, with structures based on  \ac{HP}.\cite{Tang2016} These compounds are similar to the \acp{HC}, but contain only one aromatic ring. Solid state fluorescence in single-benzene emitters is a rarity due to low melting points, but these systems showed extremely high fluorescence activity, surpassing the parent \ac{HC} systems, as shown in Table \ref{table: chalcones}.

\begin{table}[t]
\centering
\caption[Molecular structures and their \ac{QE} in the solid state]{Molecular structures of the \textbf{HC} and \textbf{HP} systems and their \ac{QE} in the solid state.\cite{Cheng2015,Tang2016,Cheng2016} } 
  \label{table: chalcones}
  \begin{tabular}{llllllll}
  %\hline
  \multicolumn{4}{c}{
  \includegraphics[height=2.7cm]{1Intro/HC.pdf}}
  & 
  \multicolumn{4}{c}{
  \includegraphics[height=2.7cm]{1Intro/HP.pdf}}\\
  \hline
  %\multicolumn{4}{c}{\textbf{HC}}&
  %&\multicolumn{4}{c}{\textbf{HP}}\\
  & R\textsubscript{1}
  & R\textsubscript{2}
  & \ac{QE}
  &
  & R\textsubscript{3}
  & R\textsubscript{4}
  & \ac{QE}\\
  \hline
  \textbf{HC1} & \ce{H} & \ce{H} & 0.32
  & \textbf{HP1} & \ce{H} & \ce{H} & 0.74\\
  %\hline
  \textbf{HC2} & \ce{CH3} & \ce{H} & 0.25
  & \textbf{HP2} & \ce{F} & \ce{H} & 0.84\\
   % \hline
  \textbf{HC3} & \ce{OCH3} & \ce{CH3} & 0.26
  & \textbf{HP3} & \ce{H} & \ce{OCH3} & 0.77\\
   %  \hline
  \textbf{HC4} & \ce{H} & \ce{CH3} & \textless0.01
  & \textbf{HP4} & \ce{H} & \ce{F} & 0.72\\
  %\hline
  \textbf{HC5} & \ce{H} & \ce{OCH3} &\textless0.01 & & & &\\
 % \hline
  \textbf{HC6} & \ce{F} & \ce{H} &0.41 & & & &\\
 % \hline
  \textbf{HC7} & \ce{H} & \ce{F} &0.10 & & & &\\
  \hline 
  \end{tabular}
\end{table}

For the \ac{HC} and \ac{HP} families, the crystal packing, absorption and emission wavelength, and crucially \ac{QE}, all depend on the choice of substituent and number of aromatic rings. How these factors interplay is not well resolved. To fully understand the photophysical properties of these systems, intricate knowledge of the electronic, molecular picture must be combined with the intermolecular interactions of the crystal. Theoretical methods can help elucidate this picture and offer insight into the working mechanisms behind \ac{AIE} for these \ac{ESIPT} systems, and how to maximise the \ac{QE}. This is the primary aim of the work in this thesis, with \ac{HC} and \ac{HP} families used as exemplars. In this thesis, Chapters \ref{chapter:NRdecay} and \ref{chapter: Inter} shall focus on the \textbf{HC} derivatives, with particular attention on \textbf{1} and \textbf{5}. In Chapter \ref{chapter: Connecting}, a holistic view of all eleven systems shall be taken as we examine how the crystal structure and packing regime influences emission. Since the \acp{HC} are the main focus of the thesis, the following section shall focus on their luminescent properties and previous investigations.  
%%%%%%%%%%%%%%%%%%%%%%%%%%%%%%
\section{Emitters Based on 2'-hydroxychalcones}\label{section: lom HC}
%%%%%%%%%%%%%%%%%%%%%%%%%%%%%%

%%%%%%
\subsection{Crystalline Emission Properties of \acp{HC}}
%%%%%%
In 2015, Cheng \textit{et al.} synthesised a range of crystalline \textbf{HC} systems with different substitution patterns.\cite{Cheng2015} They found the identity and the position of the substituent to be critical in the \ac{QE} of the crystals. This is summarised in Figure \ref{figure: HC_experimental}. When substituents are \textit{meta} to  the hydroxyl group (compounds \textbf{1}-\textbf{3}) in the phenol ring, deep red fluorescence is observed, but only when in crystalline form. The solutions are almost non-emissive. Interestingly, under frozen conditions the solutions still only weakly fluoresce, indicating that restriction of intramolecular rotation is not the key factor in the \ac{AIE} for these molecules. When the same substituents are in \textit{para} position (compounds \textbf{4},\textbf{5}), neither the crystals nor the solutions are emissive. 
\begin{figure}[H]
\centering
  \includegraphics[width=0.95\linewidth]{1Intro/HC_experimental.pdf}
  \caption[Emission behaviour of crystalline 2'-hydroxychalcone derivatives]{Compounds \textbf{1}-\textbf{5} synthesised by Cheng and coworkers. \textbf{1}-\textbf{} show high \ac{QE}, but \textbf{4}\&\textbf{5} are almost non-emissive. Absorbance maxima ($\lambda$) and \acf{QE} are given, along with pictures of the dark and bright crystals for \textbf{1},\textbf{2},\textbf{4}, and \textbf{5}. Figure adapted from ref.~\citenum{Cheng2015} with permission of Wiley-VCH.}
  \label{figure: HC_experimental}
\end{figure}
These puzzling characteristics are attributed to both the planarity of the individual molecules and the packing in the crystal, as shown in Figure \ref{figure: HC_stacking}.  Molecules \textbf{1}-\textbf{3} are almost completely planar and the strong intramolecular H-bonds, of length 1.753-1.780 \si{\angstrom}, increase the rigidity and planarity of the system. In the crystal, the molecules adopt an edge-to-face, herringbone packing mode preventing intermolecular $\pi$-interactions between the aromatic rings and enabling fluorescence from the keto S\textsubscript{1} state. For compound \textbf{4}, where a methyl substituent is \textit{para} to the hydroxyl, a similar edge-to-face packing mode is present. However, the molecule has a larger dihedral angle than \textbf{1}-\textbf{3}, which the authors attribute as the reason for the weak fluorescence. For \textbf{5}, the molecule is planar but the packing is face-to-face with aromatic stacking interactions. In the discussion, it is asserted that \textbf{5} is therefore non-emissive in the solid state due to excimer formation and non-radiative decay. The authors also hypothesise that the edge-face packed crystal of compound \textbf{4} shows minimal fluorescence because the molecule is not planar, despite other edge-face packed molecules brightly fluorescing, while the planar molecule does not emit because it is packed cofacially.
\begin{figure}[t]
\centering
  \includegraphics[width=0.95\linewidth]{1Intro/HC_stacking-screenshot}
  \caption[Crystal structures of 2'-hydroxychalcone derivatives]{The conformation of monomers \textbf{1},\textbf{4}, and \textbf{5}, along with their crystal structure. Crystal structures obtained from CCDC database codes from ref.~\citenum{Cheng2015}.}
  \label{figure: HC_stacking}
\end{figure}
It is with these hypotheses that this thesis begins. The first fundamental question is why do none of the five compounds fluoresce in good solvent? Secondly, in the solid state, why do only compounds \textbf{1}-\textbf{3} exhibit \ac{AIE}? Is this a substituent effect, as a result of the position of the methyl and methoxy groups, or is it due to the molecular packing mode, or is it a combination of both? Quantum chemical methods can identify the nonradiative decay channels in both dispersed and aggregated states, and therefore can elucidate the discrete electronic and intermolecular factors. In answering these questions, we can build understanding of how structure-property relationships operate in the \ac{AIE}/\ac{ESIPT} space and provide strategies for further optimising the fluorescence quantum yield of such systems.
%%%%%%%%%%%%%%%%%%%%%%%%%%%%%%
\subsection{Further Investigation into \textbf{HC} Photochemistry}\label{section: lom HC spectroscopy}
%%%%%%%%%%%%%%%%%%%%%%%%%%%%%%
Research into \acp{HC}, and chalcones in general, has traditionally focused on their metabolite character, since they are \textit{in vivo} precursors for a variety of flavones, flavonols, isoflavones, anthocyanidins, and other synthetic antioxidants.\cite{Singh2014} The \ac{ESIPT} process in \ac{HC} was first proven by Chou \textit{et al.} in 1992.\cite{Chou1992} With an absorption band at 354 nm, and emission at 635 nm, the authors concluded that tautomerisation occurs upon absorption and that the keto S\textsubscript{1} state was responsible for emission. Cis-trans isomerism about the central double bond followed by molecular oxygen incorporation produces the flavanoid studied by Kasha, 3-hydroxyflavone. Previous studies had investigated this cyclisation mechanism, where it was initially thought cyclisation occurred through cis-trans isomerisation without proton transfer.\cite{Stermitz1975,Matsushima1985} It was later found that this isomerism could be hindered by solvent viscosity.\cite{Tokumura1998} Later, Arai and coworkers investigated the photochemistry of \textbf{HC} analogues, where they denoted the tautomerisation to be hydrogen atom transfer, rather than proton transfer.\cite{Arai1997,Norikane2002,Norikane2003,Kaneda2003,Kaneda2003a,Kaneda2004,Teshima2009} They mapped the potential energy surfaces using spectroscopic techniques to find that the cis-trans isomerism takes place in the triplet state, and only after hydrogen atom transfer. Most pertinent was their study of the effect of substituent on fluorescence in 2009.\cite{Teshima2009} Studying systems solvated in benzene, they discovered that the addition of a methoxy substituent \textit{meta} to the hydroxyl group in the phenol ring increased the quantum yield of red fluorescence by at least ten times, which the authors attribute to a combination of electronic and steric effects. The methoxy group stabilises the S\textsubscript{1} state through electron donation to the carbonyl \pipistar orbital whilst its size hinders the intramolecular modes. They supplemented this work by studying a range of \textbf{HC} analogues with naphthol, pyrrole and indole substitutions.\cite{Shinozaki2018}

Since the discovery of the crystalline near-IR fluorescence by Cheng \textit{et. al} in 2015, and over the course of this project, more studies have emerged investigating photochemistry of \textbf{HC}s. %\textbf{HC} compounds have been developed for sensing and imaging applications.%Alkaline phosphotase is a dephosphorylating  enzyme, high concentration of which acts as a biomarker for diseases such as hepatitis, prostate cancer, and bone issues. By replacing the hydroxyl group of \textbf{HC} with a phosphoric acid group, the emission is yellow in aqueous conditions. In the presence of alkalane phosphotase, the cleave of phosphate and yields 2'-hydroxychalcone, which undergoes \ac{ESIPT} and emits in the red region.
In early 2017, Li and coworkers synthesised three analogous of \textbf{HC1} with differing substitution at the nitrogen and phenol oxygen.\cite{Li2017a} The AIE behaviour is attributed to restricting the access to the nonfluorescent \ac{TICT} state in the aggregate. Two of the systems display excellent quantum yields (0.27, 0.49) and long fluorescence lifetimes in the excited state. Time resolved spectroscopy suggests two decay paths in solid state, but the mechanisms are not identified by the authors. The near-IR technology was used to develop a probe for cysteine, as was done previously by the Tang group using a \textbf{HC} system.\cite{Song2014} 

The \ac{AIE}/\ac{ESIPT} principle was applied using \textbf{HC} to develop a technique to detect latent fingerprints, for example in crime scenes.\cite{Jin2015} A fingerprint contains a high level of sebum, and when this is rinsed a \textbf{HC} solution, the \textbf{HC} molecules preferentially adhere to the fatty acid residues in the fingerprint, where they aggregate. When light is shone on the fingerprint, \ac{ESIPT} occurs and red fluorescence is produced, lighting up the fingerprint region against the dark backdrop of the substrate.  

In late 2017, the tunable lasing properties \textbf{HC}s was further attested.\cite{Li2017d} With a fluoro subsituted \textbf{HC} (\textbf{HC6}), colour-tuned organic lasers were developed. It was shown that addition of a second hydroxyl group, $\beta$ to the carbonyl group on the aliphatic bridge, enables tuning of the photodecay. In solution, both systems undergo ESIPT and have dual emission characteristics in cylcohexane, but no ESIPT is witnessed in isopropanol, due to the formation of intermolecular hydrogen bonds. In the solid state, the compound with additional hydroxyl group does not undergo ESIPT and has blue-shifted fluorescence with respect to the parent \textbf{HC6}. The crystal structure shows that the close packing enables intermolecular hydrogen bonds, just like in isopropanol, and ESIPT is inhibited. As such, the additional hydroxyl group produces only \Estar{} emission at 538 nm, and the unmodified system shows solely \Kstar{} emission at 647m. 

More fundamental spectroscopic studies have also been presented recently. Using time-resolved absorption and emission, Zahid \textit{et. al} study the dynamics of the ESIPT process in \textbf{HC1} in both solution and the solid state.\cite{Zahid2017} In methanol, the \Estar{} state is formed and immediately decays, with only trace amounts of the \Kstar{} tautomer detected, perhaps on account of solvent molecules forming intermolecular hydrogen bonds and disrupting ESIPT. In the crystal, during the decay of the \Estar{} peak, a new fluorescence peak forms at 600-750 nm which can be ascribed to the \Kstar{} state after ESIPT. The \Estar{} form is more transient in the crystal, with its lifetime reduced by a factor of 10 to 3.1 ps, on account of the sterically-enforced planarity enabling efficient ESIPT. Enol emission will also result in self-absorption in \textbf{HC1}.

Song and co-workers studied the deactivation of \textbf{HC} as a function of solvent polarity using time-resolved absorption spectroscopy.\cite{Song2018} Increasing the solvent polarity stabilises the planar \Estar{} tautomer, due to the intramolecular charge transfer nature of the excitation. The planar \Estar{} state then promotes ESIPT. The authors find no evidence for triplet states in the three solvents. In the nonpolar solvent (cyclohexane), enol decay through intramolecular rotation will compete with ESIPT. For a set of analogues based on \textbf{HC3}, but with removal of dimethylamine group, Serdiuk \textit{et. al} found that ESIPT is more efficient than for the parent \textbf{HC} compounds, and attribute radiative decay in the solid state to the RACI model.\cite{Serdiuk2018}

The \acp{HC} and \acp{HP} systems provide a library of related compounds with differing photobehaviours. In Chapters \ref{chapter:NRdecay}-\ref{chapter: Connecting}, we shall investigate these in detail and establish design rules to optimise the fluorescence response of \ac{ESIPT} systems. However, next we shall overview the quantum chemical methods which will be used to explore the properties of these systems.


\begin{comment}
%%%%%%%%%%%%%%%%%%%%%%%%%%%%%%
\section{Structure of thesis}\label{section: lom outline}
%%%%%%%%%%%%%%%%%%%%%%%%%%%%%%
The remaining chapters of the thesis are structured as follows. In Chapter \ref{chapter:theory}, the fundamental principles of quantum chemistry are introduced, with an overview of the computational methods which have been developed in attempt to solve the \schro{} equation. The key concepts of photochemistry are discussed with a focus on excited states in molecular crystals.

In Chapter \ref{chapter:NRdecay}, we present a study of the vacuum photochemistry of five \ac{HC} compounds synthesised by Cheng and coworkers (Figure \ref{figure: HC_experimental}). Here the vertical excitations, \acp{PES} and nonadiabatic dynamics of the family are studied to determine the nonradiative decay pathways and rationalise the nonemission of these compounds.

In Chapter \ref{chapter: Inter} the focus moves to the crystalline photochemistry of \textbf{HC1} and \textbf{HC5}. Using a hierarchy of QM:MM models, we construct the \ac{PES} and the decay pathway in the excited state for each system using density functional and multireference methods. It is found that the accessible \ac{MECI} in \textbf{HC5} is responsible for nonemission in the solid state, while a similar \ac{MECI} for \textbf{HC1} is too high in energy to be populated, and thus the AIE phenomenon in \textbf{HC1} can be attributed to the RACI interpretation. The methods employed consider exciton coupling, long and short range electrostatics, and cluster size, to decouple the different factors responsible for the observed photobehaviour. Based on the calculated relaxation mechanism, we establish design rules for efficient ESIPT emitters 

In Chapter \ref{chapter: Connecting}, the scope of our work is expanded to consider the \ac{HP} family, as well as two additional \textbf{HC} compounds which undergo AIE (Table \ref{table: chalcones}). In total we consider eleven compounds to understand the remarkable efficiency of the \ac{HP} family and to test the design rules established in Chapter \ref{chapter: Inter}. A quantitative description of the crystal morphologies is developed, based upon the inherent dimer configurations. The thesis concludes with an overview of the main results, and how these precipitate the design rules for more efficient organic luminescent materials. 
\end{comment}