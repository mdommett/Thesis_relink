\begin{abstract}
%\chapter*{Abstract}
Aggregation-induced emission (AIE) offers a route for the development of solid state organic luminescent technologies, overcoming the common and undesirable phenomenon of aggregation caused quenching. Excited-state intramolecular proton transfer (ESIPT) is an attractive feature to incorporate into the an AIE-active material, which results in red-shifted fluorescence and reduced self-absorption. ESIPT coupled to AIE can produce materials with emission across the visible spectrum, with applications in imaging, detection, optoelectronics, and solid state organic lasers. However,  maximising  fluorescence  is  a formidable challenge in attaining first-principles materials design, due to the interplay between the electronic structure of the chromophore and the crystalline environment.

In this work, computational methods are used to investigate how the molecular properties and the environment mediate fluorescence for ESIPT systems. We concentrate on a family of systems, 2'-hydroxychalcones, with substituent- and morphology-dependent fluorescence. By initially isolating molecular properties, we find the systems are non-fluorescent in vacuum due to nonradiative decay \textit{via} conical intersections. Using cluster models, we then probe the potential energy surfaces in the solid state, assessing how intra- and intermolecular processes dictate fluorescence. Based on our calculations, we establish guiding principles which mediate fluorescence in these materials.

The scope is then extended to a related set of molecules, 2-hydroxyphenylpropenones, whose AIE behaviour is even more pronounced. We account for their remarkable photochemical properties through the design rules established for the 2'-hydroxychalcones. We systematically investigate competing excited state decay channels in a total of eleven systems to evaluate the factors needed for efficient ESIPT fluorophores, accounting for the crystal morphology, exciton coupling, and exciton hopping rates. This study of structure-property relationships for luminophores based on the ESIPT mechanism bridges the understanding of molecular photochemistry with crystal structure, aiding the development of highly efficient solid state emitters.
\end{abstract}