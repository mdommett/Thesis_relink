\chapter{Conclusions and Outlook}
\label{chapter: Conclusions}
%%%%%%%%%%%%%%%%%%%%%%%%%%%%%%
%\section{Overview}\label{section: conclusions_overview}
%%%%%%%%%%%%%%%%%%%%%%%%%%%%%%
We have presented an in-depth, holistic study of the photochemistry for a total of eleven ESIPT chromophores, with results which can be extended to design of more efficient luminescent organic materials. In 2015, at the start of this project, our interest was piqued by the work of the Zhang group and the unusual behaviour of their synthesised crystals.\cite{Cheng2015} We were intrigued by their hypothesis that the AIE behaviour of \textbf{HC1}-\textbf{3}, and the complete lack of fluorescence in \textbf{HC4} \& \textbf{5}, was down to molecular packing and molecular planarity alone. The electronic properites of the different substituents, in particular the electronic donating power of the methoxy group and its position of the phenol ring, seemed like a potentially important factor. The \textbf{HC} family, of five systems at that point, seemed a perfect place to begin exploring the relationship between the electronic structure and crystal morphology.

The first step was to initially explore the \acp{PES} of the five systems in vacuum. This allowed for an isolated analysis of the effect of subsituents on the parent \textbf{HC1}. We uncovered two decay pathways in vacuum which are open to the five systems, where relaxation can occur either in the \Estar{} state, or \textit{via} ESIPT to the \Kstar{} state through intramolecular rotation. In both channels, an energetically accessible \ac{MECI} can funnel population nonradiatively back to the ground state, where \ac{GSIPT} completes the four-level photocycle. The substituents determine the population of the enol and keto channels, which we investigated through \ac{TSH} dynamics for \textbf{HC1} and \textbf{HC5}, which showed the biggest deviation in absorption and emission energies in the static calculations. Due to the increased electron density loss from the phenol oxygen on account of the methoxy group, in \textbf{HC5} the rate of ESIPT is increased and the lifetime of the \Kstar{} state is reduced with respect to \textbf{HC1}. This clear electronic effect between the two systems encouraged us to study their photochemisty in the solid state.

In Chapter \ref{chapter: Inter}, we present a study of the \ac{PES} in the solid state for \textbf{HC1} and \textbf{HC5}. We use QM:MM models, with both electrostatic and mechanical embedding, with variable region size, and Ewald embedding, to explore as fully as possible the intertwined interactions present in the molecular crystal, considering ground and excited state minima, as well as conical intersections. The inherent electronic structure differences between the ground and excited state for ESIPT systems, coupled with the large intramolecular rotation in vacuum, led us to investigate the \ac{AIE} through the RACI model, rather than the \ac{FGR-RIM}. We find that in \textbf{HC1}, an excellent absorption energy is predicted and that dual emission from both \Estar{} and \Kstar{} minima is possible. While the \Estar{} emission is expected to be mostly self-absorbed, the \Kstar{} emission is somewhat overestimated with respect to experiment.%Many factors could be attributed to this, most notably the lack of mutual polarisation between the ONIOM regions and the freezing of the environmental MM atoms. This leads to under-stabilisation of the excited state and potentially does not give the conformational flexibility for the conformation to fully relax. 
Crucially, we find that the \textbf{HC1} is able to fluoresce because the \ac{MECI} lies above the excitation energy. Conversely, \textbf{HC5} can be expected to be dark because the \ac{MECI} is energetically accessible. Other deactivation mechanisms are possible in \textbf{HC5}, for instance delocalisation in the E* state. However, the propensity for ESIPT means that the RACI model is expected to dominate. 

Stepping away from the specific mechanisms in the \textbf{HC} compounds, the effect of the electronic structure and the crystalline environment have on the \ac{PES} is of general interest. Vacuum calculations at the obtained QM:MM geometries show that the energy of the MECIs are inherent to the chromophore itself. The crystalline environment provides the intermolecular interactions which prevent relaxation to the vacuum minima and \acp{MECI}, and can modulate the total energy, but their relative stability is an electronic factor. The localisation of the excited state is key to this.

We hypothesised that the more efficient ESIPT luminophore could be obtained by combining the properties of \textbf{HC1} and \textbf{HC5}. A deactivation route in \textbf{HC1} is the stability of the \Estar{} state, where fluorescence or exciton hopping can occur. If the propensity for ESIPT in \textbf{HC5} could be incorporate, the population of the \Kstar{} channel would increase and more fluorescence would be witnessed. Alternatively, in \textbf{HC5}, making the conical intersection less accessible by preventing the pyramidalisation of the carbonyl by introducing fused rings to the molecular structure.

In 2016, the Zhang group synthesised a set of ESIPT  materials with similar structural characteristics as the \textbf{HC}s. The \acp{HP} show AIE with exceptional quantum efficiencies in the solid state. In Chapter \ref{chapter: Connecting}, we expanded the scope of our work to examine the similarities and differences of the \textbf{HC}s and \textbf{HP}s. For this, we took a more quantitative approach to studying the crystalline morphology. Dimer relationships are known to important for charge and energy transfer, and so we created density maps based on the configuration of dimers found in each of the molecular crystals. The algorithms for searching and classification of dimers and calculation have been implemented into \texttt{fromage}, where they can also be useful in describing the dynamics of the early stages of aggregation.\cite{fromage} For the data in this thesis, a specific algorithm was written to quantify dimers based upon the molecular structure of the \textbf{HC} and \textbf{HP} systems, namely by locating molecular axes through the carbonyl groups. This has been generalised in the \texttt{fromage} implementation, where a plane is fitted to the molecular geometry and angles between planes are calculated in a dimer, thus meaning any molecular species can be quantified. Work in this area is ongoing in the group, for example to track dimer configurations during aggregation in molecular dynamics.  

In the \textbf{HC} and \textbf{HP} families we calculated the exciton couplings for 121 dimers and, in combination with the reorganisation energies, calculated the exciton hopping rates within the Marcus framework. We found exciton delocalisation to be far more prevalent in the \textbf{HC}s than the \textbf{HP}s, were the lack of a stable planar \Estar{} minimum promotes ESIPT more readily. Once in the \Kstar{} channel, multireference QM:MM methods show that the MECI is even more inaccessible than for \textbf{HC1}. The radiative rates were estimated by simulating emission spectra. The combination of efficient population transfer to \Kstar{} and the inaccessible MECI could be responsible for the enhanced quantum efficiencies, as was hypothesised. Modelling xciton diffusion and dynamics would be an interesting extension for these systems, where Monte Carlo simulation could harness the stochastic nature of these processes.

In Chapter \ref{chapter: Connecting} we also considered the \ac{FGR-RIM} interpretation by calculating the \ac{HR} factors in vacuum and the solid state. The large torsional rotation (90\degree{}) and the anharmonic excited state \ac{PES} results in gigantic \ac{HR} factors in vacuum for \textbf{HC1} and \textbf{HP1}, which are of course reduced significantly in the solid state where the conformation remains planar. As such, the applicability of the \ac{FGR-RIM} model is questionable for these systems. Moreover, even for \textbf{HC5}, there is a reduction of the \ac{HR} factors, yet no AIE is witnessed. Calculating the nonradiative decay rates was outwith the scope of our work, but would be an interesting avenue to further explore.

Another such avenue is the nonradiative decay in both vacuum and solid state of \textbf{HC4}, which experimentally is attributed to the nonplanarity of the chromophore. However, the conclusions we have established based on \textbf{HC5} show that there is perhaps a similar nonradiative decay funnel present. The study of the solid state \ac{PES} of \textbf{HC4} would be an interesting case in future. Also of interest is to specifically analyse the packing mode effect by arranging molecules of \textbf{HC1} in the packing arrangement of \textbf{HC5}, and determining the change in the \ac{PES}. 

The modelling photochemistry in molecular crystals is a relatively young but rapidly developing field. Sitting somewhere in between molecular and materials chemistry, it is an area ripe for exploration. The inherent difficulties and intricacies involved in modelling excited state processes are increased when incorporating the environment, with the current state of the art methods building on developments in other fields. Progress into methods to model these processes are in development, mainly through different embedding models. On this part, our group is active in pursuing this research avenue, and while more standard QM:MM methods have been used in the majority of this thesis, new developments such as the Ewald scheme have been applied to gain a more sophisticated understanding into such processes. Further developments are ongoing on these methodologies. We are implementing self-consistent polarisation schemes within QM:QM' frameworks, which work across electronic structure codes and with different QM approximations, based on multireference and density functional methods.

At the start of this work, we wanted to understand the interplay between the luminophore and the environment. The results presented in Chapters \ref{chapter:NRdecay}-\ref{chapter: Connecting} go some way in doing that, with a focus on the ESIPT phenomenon. At this point, we can conclude that incorporation of the following features into the next generation of ESIPT emitters can move the field forward:
\begin{enumerate}
    \item To maximise the population of the ESIPT channel, chromophore design should encourage a highly labile proton, where electronic excitation destabilises the \Estar{} state
    \item Packing modes should limit $\pi$-$\pi$ interactions, which enable exciton coupling and delocalisation. However, if the ESIPT is favourable enough, the localisation inherent to the \Kstar{} state can overcome unfavourable stacking arrangements.
    \item In the \Kstar{} form, solid state conical intersections are accessed \textit{via} a combination of pyramidalisation and rotation. This can be made more unfavourable by tethering and chromophore design, where we have shown that simple probes such as scanning coordinates in vacuum can predict \Kstar{} stability.
\end{enumerate}

Going forward, we hope to realise these findings by working with experimental groups to synthesise systems obeying these rules, further developing the field of luminescent organic materials for the next geneation of emissive technologies.






